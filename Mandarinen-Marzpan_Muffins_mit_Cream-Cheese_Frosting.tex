\chapter{Gebäck}
\section{Muffins}
\recipe{Mandarinen-Marzipan-Muffins mit Cream Cheese-Frosting}
\ingred{
\begin{tabular}{rp{.75cm}l}
1 & Dose & Mandarinen (175 g Abtropfgewicht)
100 & \gram & Marzipan \\
200 & \gram & Mehl \\
2 & TL & Backpulver \\
100 & \gram & Zucker \\
1 & Pckg & Vanillezucker \\
1 & Pckg & Orangenaroma \\
200 & \gram & weiche Butter \\
2 & & Eier
\end{tabular}
}

\begin{enumerate}
 \item Mandarinen in einem Sieb abtropfen lassen. Marzipan in kleine Stücke schneiden oder reißen. Die Muffinform fetten oder mit Papierförmchen füllen.
 \item Den Backofen auf $200~\celsius$ (Ober-/Unterhitze), $180~\celsius$ (Heißluft) vorheizen. 

 \item Das Mehl mit dem Backpulver in einer Schüssel vermischen. Übrige Zutaten, bis auf die Mandarinen und Marzipan, hinzufügen und mit Mixer mixen – erst auf niedriger Stufe und dann auf höchster Stufe ca. 2 Minuten lang. Dreiviertel der Mandarinen und alle Marzipanwürfel zum Teig dazugeben und vorsichtig unterheben. Sollte der Teig zu dick sein, einfach mit etwas Milch verdünnen. Den Teig in die Muffinformen geben und im Backofen ca. 25 Minuten backen lassen. Nach dem Rausnehmen noch ca. 10 Minuten in der Form ruhen lassen.
 
\end{enumerate}
\recipe{Cream Cheese Frosting}
\ingred{
\begin{tabular}{rp{.75cm}l}
150 & \gram & Weiße Kuvertüre \\
200 & \gram & Frischkäse \\
2 & & Eier \\
100 & \gram & Weiche Butter \\
bei Bedarf & & Speisefarbe
\end{tabular}
}
\hspace{0em}\\
\begin{enumerate}
 \item Für das Frosting die Kuvertüre schmelzen (ich geb die kleingehackte Schokolade in eine Schüssel und stelle sie unten in den Backofen rein, aber vorsichtig, alle 2 Minuten rühren und schauen, kann schnell anbacken.). Die geschmolzene Schokolade dann etwas abkühlen lassen. 
 \item Frischkäse mit einem Mixer cremig rühren, dann die weiche Butter und die Kuvertüre hinzufügen und weiterrühren, bis eine cremige Masse entstanden ist. Das Frosting kann jetzt nach belieben mit Speisefarbe eingefärbt werden. 
Das Frosting dann noch eine gute halbe Stunde im Kühlschrank kühlen lassen, damit sie etwas fester wird. Dann entweder mit einem Löffel auf die Muffins verteilen oder mit einem Spritzbeutel die Muffins dekorieren. 
\end{enumerate}
