\chapter{Gebäck}
\section{Cupcakes}
\recipe{Cupcakes der Versuchung}
40 Cupcakes \hfill Zubereitungszeit: $\approx2 \hour$
\ingred{
	\begin{tabular}{rp{.75cm}l}
		350 & \gram & Kakaopulver \\
		180 & \milli\litre & heißes Wasser \\
		1 \fr{1}{2} & \kilo\gram & Weizenmehl \\
		1 & TL & Backpulver \\
		1 & TL & Backnatron \\
		1 & TL & grobes Salz 
	\end{tabular}
	\hfill
	\begin{tabular}{rp{.75cm}l}
		300 & \gram & Butter \\
		500 & \gram & Zucker \\
		4 & & Eier \\
		1 \fr{1}{2} & EL & Vanilleextrakt \\
		120 & \milli\litre & saure Sahne \\
		100 & \gram & Schokotropfen
	\end{tabular}
}

\begin{enumerate}
 \item Den Ofen auf $180~\celsius$ vorheizen. Papierförmchen in Muffin-Blech einlegen.
 \item Kakaopulver und heißes Wasser in einer Rührschüssel zu einer glatten Masse anrühren.
	In einer zweiten Rührschüssel Mehl, Backpulver, Backnatron und Salz vermischen.
	Beide Schüsseln zur Seite stellen.
 \item Butter in einer Kasserolle auf mittlerer Hitze schmelzen, zusammen mit Zucker noch auf der Platte verrühren und in eine
	Rührschüssel umfüllen.
	Die Masse auf mittlerer Geschwindigkeit für etwa 4--5~\minute\ verrühren mit einem Mixer kaltrühren.
	Ein Ei nachdem anderen hinzugeben, immer eins alleine einrühren bis es komplett in der Masse gelöst ist.
	Vanilleextract und Kakaomischung mit der Masse verrühren.
	Die Geschwindigkeit des Mixers senken.
	Mehl in zwei Schüben der Masse im Wechsel mit der sauren Sahne hinzugeben.
 \item Den Teig gleichmäßig auf dem Muffinblech aufteilen (jedes Töpfchen etwa zu einem \frtext{3}{4} füllen).
	Bei $180~\celsius$ für $20 \minute$ backen (ggf. mit Zahnstocher testen).
	Nach dem Backen für etwa $15 \minute$ abkühlen lassen.
 \item Nach dem Abkühlen Glasur auftragen (empfohlen: Schokoladen-Ganache) und in den Kühlschrank stellen.
\end{enumerate}
\recipe{Schokoladen-Ganache}
reicht für 40 Cupcakes \hfill Zubereitungszeit: $\approx20 \minute$
\ingred{
	\begin{tabular}{rp{.75cm}l}
		500 & \gram & zartbittere Schokolade (geraspelt) \\
		550 & \milli\litre & Sahne (30 \%) \\
		50  & \milli\litre & Sirup (Maissirup, Zuckerrübensirup, Ahornsirup)
	\end{tabular}
}
\hspace{0em}\\
\textit{Diese Ganache dient als Glasur für Cupcakes.}
\begin{enumerate}
 \item Schokolade in einer großen, hitzefesten Schüssel ausbreiten.
 \item Sahne und Sirup bei mittlerer Hitze leicht zum kochen bringen.
	Das Gemisch über die Schokolade gießen und stehen lassen, bis die Schokolade schmilzt.
 \item Von der Mitte aus beginnend die geschmolzene Schokolade mit der Sahne verrühren, bis sie glatt ist (nicht zu lange rühren).
 \item In den Kühlschrank stellen und alle $5~\minute$ rühren, bis die Creme von alleine die Form behält und etwas aufhellt.
	Sofort auftragen (Ganache wir dicke, wenn man zu rühren aufhört)
\end{enumerate}
