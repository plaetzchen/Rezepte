\chapter{Gebäck}
\section{Kekse}
\recipe{American Cookies}
30 Kekse \hfill Zubereitungszeit: $\approx35 \minute$
\ingred{
	\begin{tabular}{rp{.75cm}l}
		200 & \gram & weiche Butter \\
		200 & \gram & weißer Zucker \\
		200 & \gram & brauner Zucker \\
		2 &  & große Eier \\
		1 & Pckg & Vanillezucker \\
		\fr{1}{2} & TL & Salz \\
		1 & Pckg & Backpulver \\
		300 & \gram & Mehl \\
		200 & \gram & Zartbitterschokolade \\
	\end{tabular}
}

\begin{enumerate}
\item	Butter und beide Zuckersorten schaumig rühren.
	Die Eier nacheinander unterrühren.
	Vanillezucker zugeben.
\item	In einer anderen Schüssel Salz, Backpulver und Mehl mischen.
	Die Buttermischung zufügen und unterrühren.
	Währenddessen die Schokolade zugeben.
	Wenn der Teig sehr weich ist.
	Für $30~\minute$ abgedeckt in den Kühlschrank stellen.
\item	Den Backofen auf $180~\celsius$ vorheizen.
	Mithilfe eines Esslöffels kleine Bällchen Teig abteilen und mit Abstand auf ein gefettetes Backblech setzen.
	Im vorgeheizten Backofen 8--10~\minute\ backen.
	Die Cookies sollen noch sehr weich sein!
	Herausnehmen und auf einem Kuchengitter abkühlen lassen.
\end{enumerate}

\section{Muffins \& Cupcakes}
\recipe{Cupcakes der Versuchung}
40 Cupcakes \hfill Zubereitungszeit: $\approx2 \hour$
\ingred{
	\begin{tabular}{rp{.75cm}l}
		100 & \gram & Kakaopulver \\
		200 & \milli\litre & heißes Wasser \\
		400 & \gram & Weizenmehl \\
		1 & Pckg & Backpulver \\
		1 & Pckg & Backnatron \\
		1 & TL & grobes Salz 
	\end{tabular}
	\hfill
	\begin{tabular}{rp{.75cm}l}
		300 & \gram & Butter \\
		450 & \gram & Zucker \\
		4 & & Eier \\
		1 & Pckg & Vanilleextrakt \\
		250 & \milli\litre & saure Sahne \\
		100 & \gram & Schokotropfen
	\end{tabular}
}

\begin{enumerate}
 \item	Den Ofen auf $180~\celsius$ vorheizen. Papierförmchen in Muffin-Blech einlegen.
 \item	Kakaopulver und heißes Wasser in einer Rührschüssel zu einer glatten Masse anrühren.
	In einer zweiten Rührschüssel Mehl, Backpulver, Backnatron und Salz vermischen.
	Beide Schüsseln zur Seite stellen.
 \item	Butter in einer Kasserolle auf mittlerer Hitze schmelzen, zusammen mit Zucker noch auf der Platte verrühren und in eine
	Rührschüssel umfüllen.
	Die Masse auf mittlerer Geschwindigkeit für etwa 4--5~\minute\ verrühren mit einem Mixer kaltrühren.
	Ein Ei nachdem anderen hinzugeben, immer eins alleine einrühren bis es komplett in der Masse gelöst ist.
	Vanilleextract und Kakaomischung mit der Masse verrühren.
	Die Geschwindigkeit des Mixers senken.
	Mehl in zwei Schüben der Masse im Wechsel mit der sauren Sahne hinzugeben.
 \item	Den Teig gleichmäßig auf dem Muffinblech aufteilen (jedes Töpfchen etwa zu einem \frtext{3}{4} füllen).
	Bei $180~\celsius$ für $20~\minute$ backen (ggf. mit Zahnstocher testen).
	Nach dem Backen für etwa $15~\minute$ abkühlen lassen.
 \item Nach dem Abkühlen Glasur auftragen (empfohlen: Schokoladen-Ganache) und in den Kühlschrank stellen.
\end{enumerate}
\recipe{Schokoladen-Ganache}
reicht für 40 Cupcakes \hfill Zubereitungszeit: $\approx20 \minute$
\ingred{
	\begin{tabular}{rp{.75cm}l}
		500 & \gram & zartbittere Schokolade (geraspelt) \\
		500 & \milli\litre & Sahne (30 \%) \\
		50  & \milli\litre & Sirup (Maissirup, Zuckerrübensirup, Ahornsirup)
	\end{tabular}
}
\hspace{0em}\\
\textit{Diese Ganache dient als Glasur für Cupcakes.}
\begin{enumerate}
 \item	Schokolade in einer großen, hitzefesten Schüssel ausbreiten.
 \item	Sahne und Sirup bei mittlerer Hitze leicht zum kochen bringen.
	Das Gemisch über die Schokolade gießen und stehen lassen, bis die Schokolade schmilzt.
 \item	Von der Mitte aus beginnend die geschmolzene Schokolade mit der Sahne verrühren, bis sie glatt ist (nicht zu lange rühren).
 \item	In den Kühlschrank stellen und alle $5~\minute$ rühren, bis die Creme von alleine die Form behält und etwas aufhellt.
	Sofort auftragen (Ganache wir dicke, wenn man zu rühren aufhört)
\end{enumerate}


\recipe{Mandarinen-Marzipan-Muffins mit Cream Cheese-Frosting}
\ingred{
\begin{tabular}{rp{.75cm}l}
1 & Dose & Mandarinen (175 g Abtropfgewicht) \\
100 & \gram & Marzipan \\
200 & \gram & Mehl \\
2 & TL & Backpulver \\
100 & \gram & Zucker \\
1 & Pckg & Vanillezucker \\
1 & Pckg & Orangenaroma \\
200 & \gram & weiche Butter \\
2 & & Eier
\end{tabular}
}

\begin{enumerate}
 \item Mandarinen in einem Sieb abtropfen lassen. Marzipan in kleine Stücke schneiden oder reißen. Die Muffinform fetten oder mit Papierförmchen füllen.
 \item Den Backofen auf $200~\celsius$ (Ober-/Unterhitze), $180~\celsius$ (Heißluft) vorheizen.
 \item Das Mehl mit dem Backpulver in einer Schüssel vermischen. Übrige Zutaten, bis auf die Mandarinen und Marzipan, hinzufügen und mit Mixer mixen – erst auf niedriger Stufe und dann auf höchster Stufe ca. 2 Minuten lang. Dreiviertel der Mandarinen und alle Marzipanwürfel zum Teig dazugeben und vorsichtig unterheben. Sollte der Teig zu dick sein, einfach mit etwas Milch verdünnen. Den Teig in die Muffinformen geben und im Backofen ca. 25 Minuten backen lassen. Nach dem Rausnehmen noch ca. 10 Minuten in der Form ruhen lassen.

\end{enumerate}
\recipe{Cream Cheese Frosting}
\ingred{
\begin{tabular}{rp{.75cm}l}
150 & \gram & Weiße Kuvertüre \\
200 & \gram & Frischkäse \\
2 & & Eier \\
100 & \gram & Weiche Butter \\
bei Bedarf & & Speisefarbe
\end{tabular}
}
\hspace{0em}\\
\begin{enumerate}
 \item Für das Frosting die Kuvertüre schmelzen (ich geb die kleingehackte Schokolade in eine Schüssel und stelle sie unten in den Backofen rein, aber vorsichtig, alle 2 Minuten rühren und schauen, kann schnell anbacken.). Die geschmolzene Schokolade dann etwas abkühlen lassen.
 \item Frischkäse mit einem Mixer cremig rühren, dann die weiche Butter und die Kuvertüre hinzufügen und weiterrühren, bis eine cremige Masse entstanden ist. Das Frosting kann jetzt nach belieben mit Speisefarbe eingefärbt werden.
Das Frosting dann noch eine gute halbe Stunde im Kühlschrank kühlen lassen, damit sie etwas fester wird. Dann entweder mit einem Löffel auf die Muffins verteilen oder mit einem Spritzbeutel die Muffins dekorieren.
\end{enumerate}

\recipe{Buttercream (with Cream Cheese)}
(Mein basisrezept) \hfill Zubereitungszeit: $\approx20 \minute$
% TODO
\ingred{
	\begin{tabular}{rp{.75cm}l}
		\fr{1}{2} & cup & \emph{weiche} Butter \\
		225 & \gram & Frischk"ase \\
		500  & \gram & Puderzucker \\
		& bei Bedarf & Speisefarbe
	\end{tabular}
}
\textit{Buttercream eignet sich hervorragend als Sahneh"aubchen f"ur Cupcakes etc.
Das Allerwichtigste beim Buttercreammachen: Die Butter muss weich sein. Glipschig, matschig, im-Sommer-vergessen-in-denK"uhlschrank-zu-stellen weich.
Am besten nimmt man sie also (je nach Jahreszeit und Heizverhalten) ein paar Stunden vor dem Zubereiten
aus dem K"uhlschrank. Den Frischk"ase kann man sp"ater danaben stellen, denn wenn die Temperaturdifferenz zwischen den beiden zu gro"s ist, flockt's.
(Die Mengenangaben dienen als Orientierung, ich misch meistens frei Schnauze)}

\begin{enumerate}
\item	Butter und Frischk"ase in eine R"uhrsch"ussel geben und mixen, bis man eine fluffige Masser erh"alt.
      Dann so lange Puderzucker dazugeben und verr"uhren, bis das ganze cremig und so s"u"s wie gew"unscht ist.
      Bei Bedarf Speisefarbe unterr"uhren.
\item	Ich stelle nun das ganze f"ur mindestens $30~\minute$ in den K"uhlschrank, bevor ich es in eine Spritzt"ute f"ulle
      und dann auf den (abgek"uhlten!) Cupcakes verteile. Damit das Gebilde bis zum Verzehr seine Form beh"alt,
      geht das ganze danach abermals f"ur mindestens $30~\minute$ in den K"uhlschrank.
\end{enumerate}

\recipe{Raspberry Cream Cheese Buttercream}
\hfill Zubereitungszeit: $\approx20 \minute$
This one is pure pr0n. (Denk dran, die Himbeeren ordentlich abzutauen und abzutropfen!)
% TODO
\ingred{
	\begin{tabular}{rp{.75cm}l}
		\fr{3}{4} & cup & \emph{weiche} Butter \\
		\fr{1}{2} & cup & Frischk"ase \\
		2 \fr{1}{2} & cups & Puderzucker \\
		\fr{2}{3} & cup & Himbeeren (frisch oder aufgetaut) \\
	\end{tabular}
}

\begin{enumerate}
\item	Butter und Frischk"ase in eine R"uhrsch"ussel geben und mixen, bis man eine fluffige Masser erh"alt.
      Dann \fr{1}{2} cup Puderzucker dazugeben und verr"uhren, bis das ganze cremig ist. Nun
      abwechselnd insgesamt \fr{2}{3} cup Himbeeren und 2-3 \fr{1}{2} cup Puderzucker hinzugeben,
      dabei bei mittlerer Geschwindigkeit r"uhren, bis die Buttercream fluffig-cremig ist.
\item	Ich stelle nun das ganze f"ur mindestens $30~\minute$ in den K"uhlschrank, bevor ich es in eine Spritzt"ute f"ulle
      und dann auf den (abgek"uhlten!) Cupcakes verteile. Damit das Gebilde bis zum Verzehr seine Form beh"alt,
      geht das ganze danach abermals f"ur mindestens $30~\minute$ in den K"uhlschrank.
\end{enumerate}
