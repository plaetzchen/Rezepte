\chapter{Brot}
\recipe{Ciabatta}
1 Portion \hfill Zubereitungszeit: $\approx1 \hour$ Arbeitszeit + $\approx3 \hour$ Ruhezeit
\ingred{
	\begin{tabular}{rll}
	 5 & \gram & Hefe (frisch) \\
	 450 & \gram & Mehl (Weizen) \\
	 10 & \gram & Hefe (frisch) \\
	 5 & EL & Milch \\
	 2 & EL & Oliven"ol \\
	 20 & \gram & Salz \\
	 550 & \gram & Mehl (Weizen) \\
	 Mehl \\
	"Ol \\
	Grie"s \\
	\end{tabular}
}
\begin{enumerate}
\item	Am Vortag f"ur den Vorteig die 5 \gram Hefe in 250 ml handwarmem Wasser 10 min gehen 		lassen. Das Mehl gr"undlich unterr"uhren. Zugedeckt gehen lassen.
\item   	Am n"achsten Tag f"ur den Hauptteig die 10 \gram Hefe in der lauwarmen Milch 10 min gehen lassen. Mit 250 ml lauwarmem Wasser, dem "Ol und dem Vorteig verr"uhren. Salz und 		Mehl unterkneten.
\item
Auf bemehlter Fl"ache 2-3 min kr"aftig kneten, ein"olen und abgedeckt 1,5 h bei Zimmertemperatur gehen lassen, bis der Teig sehr luftig und leicht klebrig ist.
\item
Den Teig vierteln, die Viertel zu Rollen formen und so ziehen, dass sie handbreit und 30 cm lang werden.
\item
Mit der Naht nach oben auf bemehlte St"ucke Backpapier setzen. Mit den Fingern mehrfach leicht eindr"ucken und mit bemehlten Geschirrt"uchern bedeckt 1,5-2 h gehen lassen.
\item
Einen Pizzastein (oder Terrakottaziegel) auf die zweite Einschubleiste von unten auf ein Gitter legen und den Ofen auf 220 Grad (Gas 3-4, Umluft nicht empfehlenswert) vorheizen. Den Pizzastein (oder Terrakottaziegel) mit etwas Weizengrie"s bestreuen. 2 Brote umgedreht drauflegen. (Die beiden anderen in den K"uhlschrank legen.) Das Papier entfernen. 100 ml Wasser auf den Ofenboden gie"sen, sofort die T"ur schlie"sen. 25 min backen. Auf Gittern ausk"uhlen lassen. Die beiden anderen Brote auf dieselbe Weise backen.
\end{enumerate}

Quelle: \url{http://www.chefkoch.de/rezepte/304983375443/Ciabatta.html}
